\documentclass[a4paper,twocolumn]{article}
\usepackage{graphicx}
\usepackage[utf8]{inputenc}
\usepackage[T1]{fontenc}
\usepackage[czech]{babel}
\usepackage[top=1.2cm, bottom=1.2cm, left=1.5cm, right=1.5cm,footskip=0.4cm]{geometry}
\usepackage{nopageno}
\usepackage{fancyhdr}
\usepackage{hhline}
%\usepackage{showframe}

\begin{document}

    \renewcommand{\arraystretch}{1.5}

\footnotesize

\title{BSG Cheat Sheet [CZ]}
\author{Vojtěch Rain Vladyka}
\date{Verze 1.2}

\maketitle

\section{Tah hráče}
\begin{enumerate}
\item líznout si karty schopností (skill cards)
\item pohyb 
	\begin{itemize}
	\item v rámci lodě 
	\item mezi loděmi - odevzdej jednu kartu schopnosti
	\item při pilotování viperu je možný pohyb do vedlejšího sektoru
	\item v rámci karet Pohyb (Movement)
	\end{itemize}
\item akce 
	\begin{itemize}
	\item na lokaci
	\item na kartě schopnosti
	\item speciální schopnost postavy
	\item při pilotování viperu
		\begin{itemize}
		\item let do sousedního sektoru
		\item útok
		\item použití karty schopnosti
		\end{itemize}
	\item odhalit se jako Cylon
	\end{itemize}
\item krize - líznout si kartu krize a vyřešit ji
	\begin{itemize}
	\item zkouška schopností (Skill check)
	\item Cylonský útok
	\item Událost (Event)
	\end{itemize}
\item aktivace zbylých (v tahu nehraných) raiderů (heavy raidery se nehýbou)
\item příprava na skok
\end{enumerate}

Maximální počet karet v ruce je 10.\\

\section{Zkouška schopností}
\begin{enumerate}
\item sestavení balíčku
	\begin{enumerate}
 	\item zahrát karty modifikující balíček (např. \emph{Reckless})
	\item přidat 2 karty z Osudu (\emph{Destiny deck})
	\item každý hráč popořadě od aktivního hráče přidá libovolný počet karet
	\end{enumerate}
\item vyřešení 
	\begin{enumerate}
	\item zamíchání balíčku
	\item roztřídění podle barev - barvy na kartě zkoušky a zbylé barvy
	\item požadované barvy se sečtou a odečtou se zbylé barvy
	\item je-li výsledek roven nebo vyšší než síla zkoušky, úspěch.
	\item zahrát karty modifikující výsledek
	\end{enumerate}
\item vedlejší účinky
	\begin{enumerate}
	\item Bezohledná (\emph{Reckless}) - pokud byla zkouška učiněna Bezohlednou, musí se provést případné důsledky z karet Zrady (\emph{Treachery})
	\item Pokud byla zkouška označena s důsledky (\emph{Skill Check ability icon}) a byla v ní alespoň jedna takto ozančená karta, musí se provést scénář s důsledky.
	\end{enumerate}
\end{enumerate}

\section{Událost}
Aktivní hráč, prezident nebo admirál vybírají z několika předložených variant

\section{Boj}
Kdykoli jsou na herní ploše Cylonské lodě, je flotila pod útokem.\\
\\
\begin{tabular}{ | l | p{5.5cm} | }
\hline
\textbf{Loď} & 	\textbf{Vyhodnocení} \\
\hline
Raider & 		3-8 zničený \\
Heavy Raider & 	7-8 zničený \\
Viper mk.II & 	5-7 poškozený \\
&				8 zničený \\
Viper mk.VII &	6-7 poškozený \\
&				8 zničený \\
Civilní loď	&	zničená automaticky\\
Galactica &		\begin{itemize}
				\item[] Raiderem: 8 poškozená 
				\item[] Basestar: 4-8 poškozená
				\end{itemize}
				\\
Basestar & 		Viperem 
				\begin{itemize}
				\item[] Viperem: 8 poškozená
				\item[] Galacticou: 5-8 poškozená 
				\end{itemize}
				\\
Útok atomovou hlavicí & Pro vybranou oblast:
				\begin{itemize}
				\item[] 1-2 poškodit basestar 2x 
				\item[] 3-6 zničit basestar
				\item[] 7 zničit basestar spolu s 3 raidery
				\item[] 8 zničit \textbf{všechny} lodě v oblasti
				\end{itemize}
				\\
\hline
\end{tabular}
\\\\
Zničení Galacticy - 6 a víc poškození\\
Zničení Basestar - 3 a víc poškození\\
%\pagebreak
\subsection{Aktivace Raiderů}
Pouze jedna z akcí za tah
\begin{itemize}
\item útok:
	\begin{enumerate}
	\item nepilotovaný Viper
	\item pilotovaný Viper
	\item civilní loď
	\item Galactica
	\end{enumerate}
\item let nejkratší trasou k cílům, je-li trasa stejně dlouhá, letí po směru hodinových ručiček
	\begin{itemize}
	\item když v sektoru není na koho by mohl útočit (tj. raiderů je víc jak cílů)
	\item vždy o jeden sektor
	\end{itemize}
\end{itemize}

\subsection{Aktivace Heavy raiderů}
Let
\begin{enumerate}
\item nejkratší trasou k hangáru
\item začíná-li let v sektoru s hangárem, naloďují se
\end{enumerate}

\subsection{Aktivace viperů}
Vždy jen jedna z možností
\begin{enumerate}
\item let do sousedního sektoru (Viper mk.II)
\item let do sousedního nebo do následujícího sektoru (Viper mk.VII)
\item útok
\item eskortování civilní lodě\footnote{Varianta s Cylonskou flotilou}
\end{enumerate}

\section{Tah cylona}
\begin{enumerate}
\item líznout si 2 karty schopností, každou jiného druhu
\item pohyb 
	\begin{itemize}
	\item v rámci cylonských lokací 
	\item v rámci karet Pohyb (Movement)
	\end{itemize}
\item příprava na skok
\end{enumerate}
Poznámka: neaktivují se Raidery

\section{Poprava}
\begin{enumerate}
\item zahodí všechny karty schopností z ruky a všechny přiřazené karty Kvora. Karty Kvora v ruce zůstávají nedotčené.
\item ověří se hráčova loajalita
\item Člověk
	\begin{itemize}
	\item ztráta jednoho bodu morálky
	\item hráčova postava se vrací a nesmí se už do konce hry použít
	\item hráč si vybere novou postavu
	\item pokud byla popravena před fází Spícího agenta Boomer, její hráč dostává druhou kartu loajality okamžitě
	\end{itemize}
\item Cylon
	\begin{itemize}
	\item přesun na Loď znovuzrození
	\item pokračuje normálně jako odhalený Cylon, ale nedostává Super Krizi
	\end{itemize}
\end{enumerate}

\section{Nová Caprica}
\textbf{Příprava}
\begin{enumerate}
	\item Vyměnit balíček krizí za krize Nové Capricy
	\item Přesunout lidi do \textit{Resistance HQ}, cylony do \textit{Occupation Autohority}
	\item Přesunout civilní lodě do zásobníku \textit{Uzamčených lodí}
\end{enumerate}
\textbf{Cíl}\\
Odletět z Nové Capricy. Admirál rozkáže poslední skok, všechny lodě na Nové Caprice jsou zničeny, všichni lidé zanecháni pozadu popraveni. Poté hra končí. Pokud zůstaly všechny ukazatele nad nulou, lidé vyhráli.\\
\textbf{Speciální pravidla}
\begin{itemize}
	\item Okupační síly - Cylonské okupační síly postupují k Loděnici s cílem zničit lodě. Když dojdou až k nim, otočí se vrchní a zničí ji. Okupační síly se vracejí. Může současně jít několik okupačních sil za sebou.
	\item Příprava lodí a Evakuace - Dokud se Galactica nevrátí na orbitu, lodě se pouze přemisťují z \textit{Uzamčených lodí} do \textit{Připravených lodí}. Když je Galactica na orbitě (ukazatel skoku na auto-jump), mohou se Připravené lodě přesunout na orbitu. 
\end{itemize}

\section{Cylonští vůdci}
Cylonští vůdci hrají jako odhalní Cyloni s následujícími výjimkami:
\begin{itemize}
	\item Lížou si karty dle svých schopností
	\item Musí se řídit svými schopnostmi (i negativními), mohou aktivovat lokace a akce na kartách zrady.
	\item Mohou infiltrovat Galacticu. Poté se jich týkají následující změny:
	\begin{itemize}
		\item jsou bráni jako lidé,
		\item nemohou získat Admirálský ani Prezidentský titul,
		\item lížou si jednu kartu navíc ze svých schopností,
		\item mohou dát pouze dvě karty do Zkoušek,
		\item může se kdykoli vrátit v rámci akce na Loď znovuzrození.
	\end{itemize}
\end{itemize}
\textbf{Cíl Cylonského vůdce}\\
Každý Cylonský vůdce získává kartu se svým osobním cílem. Veškeré podmínky jsou uvedeny na kartě.
\end{document}
